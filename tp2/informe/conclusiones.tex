Durante las experimentaciones se comprobó que el método para determinar la ruta
al host de destino es de naturaleza probabilística ya que hubo veces (con muy
poca probabilidad igualmente) en las que, entre dos Echo Request, para un mismo
TTL se daban varias IPs (nunca más de 3) dificultando esto la veracidad de la
ruta.

Con respecto al método de detección de Outliers de Cimbala se pudo comprobar
que este resulta bastante robusto y efectivo para una cantidad pequeña de
muestras. Aunque tanto en el experimento de las redes de las universidades, se pudo observar que
un único valor de corte, que no depende de la cantidad de muestras, establecido en cualquier número
entre 1.4 y 1.6 aproximadamente detecta exactamente los mismos outliers que el
método de Cimbala. Para solucionar los falsos positivos/falsos negativos de los
distintos resultados se concluyó que ya no es posible un único valor de corte ya
que para el de la Habana (bajo el supuesto que la verdadera ruta pasa por
Italia) como se menciono anteriormente debería ser 1.2, para el de la red de
universidades de Australia 1.9 y este a su vez también sirve para el de la
universidad de Rusia. También al observar los valores considerados outliers por cimbala
de los experimentos 1.87, 2.168, 2.30 notamos que estos tienen una media de 2.11
y una desviación estandar 0.17 (es decir un intervalo -1.94, 2.28) lo que nos
deja la pregunta abierta que si al aumentar la cantidad de experimentaciones
el promedio de los valores $Z_i$ del tiempo de enlace de los cables intercontinentales
tienen una media y una desviación similar a la calculada dándonos como aplicación
alternativa la de detectar gateways que se encuentran operando con dificultades (lentamente).

Adicionalmente durante la experimentación se pudo demostrar empíricamente como los servicio
de localización como PlotIP y GeoIPTool son poco precisos, no confiables, y contienen bastantes
errores en su base de datos para la determinación de verdaderos enlaces intercontinentales o
transatlánticos llegando a ser más preciso o algorítmico en ciertas ocasiones las mediciones de tiempo
de la ida y vuelta de los paquetes junto con el método propuesto por Cimbala.

Finalmente, pudimos entender la arquitectura de algunos sitios de la Internet gracias al protocolo \textbf{ICMP} (ie. como se llega a casi
cualquier host por ejemplo en menos de 35 saltos, los distintos nodos que actuan siempre, etc) como lo son en este caso
estas universidades o como tantos de los sitios web que están alojados en otros
continentes
